\section{Language}

Generally, the language of a scientific research text should be neutral, matter-of-factly,
and not contain colloquial wording. After all, the reader of a scientific text wishes to
receive precise information rather than to be entertained.

\begin{quotation}
  \fbox{
    \begin{minipage}{.876\textwidth}
      \subsubsection*{Example (poor)}
      
      After a short while the sender wants to post the message again.
      
      \subsubsection*{Example (better)}
      
      After 10 seconds the sender retries to transmit the message.    
    \end{minipage}
}
\end{quotation}

The "first person" should not be used in the type of elaborations discussed in
this text. In German texts it is common to use the passive voice. In English texts you will
often encounter the usage of "we".

\begin{quotation}
  \fbox{
    \begin{minipage}{.876\textwidth}
      \subsubsection*{Example (poor)}
      Therefore I decided to use solution XY.

      \subsubsection*{Example (better)}
      Therefore, we decided to implement solution XY. ("English style")
      
      or
      
      Therefore, it was decided to implement solution XY ("German style") 
      \end{minipage}
  }
\end{quotation}

Generally, in a scientific text it is obligatory to substantiate all statements
that are not obvious. This can be done by means of own research results (as, for
example, in a Master thesis), or by referring to other sources providing either
evidence or a logical and coherent motivation.

During the writing process, the author should imagine himself in the place of the
reader, making sure that he, being the reader, would still understand the text.
It is a matter of course to write in full, grammatically correct sentences and
avoid spelling mistakes. We highly recommend to use spell checkers (in LaTeX e.g.
GNU ispell \cite{ispell_homepage}) and thoroughly proof-read the text at least once
before handing it in.




