\chapter{Structure}

A very common method of starting an elaboration is the so-called \emph{top-down},
i. e. the drafting of an appropriate structure of chapters that lateron will
be refined and filled with text. The most basic structure divides the text
into introduction, main part, and conclusion.
 

\section{The Introduction}

Ideally, the introduction convinces the reader that it is worthwhile reading the
rest of the paper, i.e. it should name the tasks of the work and explain
why their investigation is of interest. In order to demonstrate the relevance
of a problem it can be helpful to present it in a larger context.

While the beginning of the introduction should sufficiently motivate the reader 
to continue reading, the end of the introduction should give a survey of the
chapters to follow, outlining in short how the task was approached.


\section{The Main Part}

The main part is the actual core of the paper and consists of several chapters
which in terms of argument and presentation should form a logical line.
Lab reports and Master theses often require the following structure, with each
point possibly leading to several chapters:


\begin{description}
  
\item[Introduction of basics] First of all, it is important to
  introduce the basic concepts of the work. It can be taken for granted
  that the reader is familiar with the contents of the lecture 
  "High Performance Networking (MA-INF 3101)" (for seminar or lab in the
  second semester) and of the lecture(s) "Network Security (MA-INF 3201)"
  and/or "Mobile Communication (MA-INF 3202)" (for seminar, lab or 
  Master thesis in the third or higher semester). By no
  means we must lapse into writing a piece of universal technical literature.
  It is important to introduce only those basics that the paper/thesis will
  lateron refer to. Additional details of no relevance for the work will just
  distract the reader's attention and not only waste space but also
  strain the reader's (and the assessor's :-)) patience.
  
\item[Specification of the task] Whereas in the introduction the task 
  was outlined rather abstractly and in a larger context, it should now be
  specified: What exactly will be examined in this work (and what will not)?
  
\item[Description of possible solutions] Drawing upon the former
  chapters, we now have to develop our own solutions. To "look beyond
  one's own nose" and document what others hitherto contributed to the
  task are essential parts of scientific work. If task-related works do exist,
  it is good to emphasize the differences between them and our own work.

\item[Assessing the methods of solution] It is easy to allege things!
  However, it needs formal evidence or an empirical proof to make a statement
  valuable. Therefore, a significant part of the work accounts for examining the
  methods of resolution. This is the place for presenting analytical, simulative,
  or prototypical measuring results in order to emphasize the quality and special
  applicability of our own method (in comparison with other approaches). The
  "road" leading from assumptions to conclusions should be comprehensible
  for the reader to an extent that enables him to repeat the research for 
  verification.

\end{description}

If the length of a chapter exceeds a certain limit, it is helpful to start with a
brief introduction that picks up the thread leading through the document and outlines
the contents of the following chapter. Also, each chapter should end with a short
summary stressing essential points of the chapter.


\section{The Conclusion}

The final chapter ("Summary") summarizes the primary aspects of the work
and present its essence. The summary should be comprehensible even for readers who
are not familiar with the contents of the main part. The reader, just by reading the
introduction and the summary, should get an idea of the document's tasks as well as
their solutions.

Also, this is the adequate place for presenting our own notions about the task in a 
short résumé.

In lab reports and Master theses, conclusion should also mention the prospects
of possible further work, especially questions that arose in the course of the
work but for reasons of time and/or space could not be dealt with in depth.
