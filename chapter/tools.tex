\chapter{Choosing the Appropriate Tool}

Prior to starting the actual writing process, the student has to decide which text
processing program he will use. The answer to this question is quite easy to give:
Basically we do not have any preference as to the tools used as long as we receive
the result as a \textbf{PDF-file}.


\section{LaTeX}

LaTeX (spoken:"Lahtech") (\cite{latex_book,latex}) sets scientific texts
in premium quality. Comparing LaTeX with other text processing programs, like MS Word
or Open Office, you will easily note the difference in quality. Also, LaTeX is highly
reliable; it will not fail even with the 100th page, the 100th reference, the 100th
mathematical formula and the 100th picture within one document.

Those who have so far worked only with WYSIWYG-tools might have to get used to
giving layout instructions by embedding the respective LaTeX commands in the text.
Also, some (few) basic structures have to be understood. Once the necessary
"working set" of LaTeX-structures is established, the composition of
further texts with LaTeX will be very easy.

On this background, it is recommended that you get used to working with LaTeX as early
as possible, for example in the course of working on the seminar paper or lab report.
A lot of excellent assistance can be found in the Web, for example on the site of
DANTE e.V. (\cite{dante}).


\section{Open Office, MS Word, etc.}

From a very simplified point of view, there are as many shortcomings to LaTeX as there
are advantages to common WYSIWYG text processing programs, such as Open Office or MS Word
(and vice versa), where it is possible to start working right away and quickly get to
an acceptably formatted result. The difficulties often start in the course of working
on long texts containing numerous pictures and formulas, since they will lead to a rapid
decrease of reliability and processing speed.


\section{Recommendation}
Basically, our experience shows that LaTeX is the best choice when working with complex
documents. The initial work input which is necessary to get used to this tool will lateron
be rewarded by high stability, a minimum of layout work, high print quality, and, last
but not least, a significant reduction of trouble.
