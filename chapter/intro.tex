\chapter{Motivation}

This guideline for composing master theses, seminar papers and lab reports was
inspired by the observation that, in the process of their work, students often
repeat the same mistakes that could easily be avoided. On this background, the
idea for this guideline was born with the intention to reduce your (and also,
of course, our) time input, and provide you with a set of techniques for composing
scientific texts which have proven very effective for improving the presentation of
contents as well as as their comprehensibility for the reader.


\section{The Purpose of a Lab Report}

Preparing the report is an inherent part of every lab offered by our work group
during the main study period. The report should give the reader a detailed picture of
 
\begin{itemize}
\item which task was tackled during the practical exercises,
\item which challenges had to be coped with in order to accomplish the task,
\item in what way and how well these challenges were mastered.
\end{itemize}

The report is \textbf{not a protocol of procedures}, i.e. it should not provide a
detailed listing of all steps made in order to solve the problem. It is rather a
documentation of the applied solution and should also motivate why this particular
solution was chosen. It can be appropriate to also mention other possible solutions
that were tried but lateron abandoned for good reasons. However, the description of
the implemented solution must definitely outweigh such references.


\section{The Purpose of a Seminar Paper}

A seminar paper should summarize in short the vital aspects of a given subject.
Since the size of the text sources usually outnumbers the admitted paper volume by
far, it is inevitable for the author to reduce the sources to the relevant facts.
The paper should be written in the author's own words and never be literally copied
from the original text. This rule particularly applies to text sources in foreign
languages: literal translations are usually easy to recognize for the simple reason
that they are difficult to read; apart from that, they simply miss the point of the
matter. One of the excuses for using literal translations is that the original text
could not be understood. If this is the case, rather consult your tutor for help --
that is what he is there for. Other popular excuses like "This source was so
excellent, I could not have said it better" certainly do not require any further comments.


\section{The Purpose of a Master Thesis}
\label{sec:aufgaben-diplom}

According to the MaPO (conditions of study) of 2008 \cite{mpo08}, the Master thesis
is defined as (original quotation from the German language MaPO):

\begin{quotation}
  "\S 17 (1) Die Masterarbeit ist eine schriftliche Pr"ufungsarbeit, die zeigen soll, 
  dass der Pr"ufling in der Lage ist, innerhalb einer vorgegebenen Frist ein Problem aus
  dem Gebiet des Studienganges selbst"andig nach wissenschaftlichen Methoden zu bearbeiten,
  einer L"osung zuzuf"uhren und diese angemessen darzustellen."
\end{quotation}

A corresponding English meaning would be:

\begin{quotation}
  "... the Master thesis is expected to show that the student is capable of 
  independently applying scientific methods 
  to a problem in the field of computer science within a set period of time,
  proving his/her aptitude for self-dependent scientific work."
\end{quotation}

The Master thesis is reviewed on the basis of the written elaboration handed in by
the student. Therefore, for the student's own benefit it is recommended to focus not
only on the contents, but also on an appealing form of their presentation. Normally,
the grade is not strongly influenced by formal aspects of the thesis. If, however,
we have to make a choice between two possible gradings, the form of presentation can
be of vital significance.
