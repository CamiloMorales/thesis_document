\chapter{Formal Basics}


\section{The Volume of a Paper}

The volume of a paper consists of the amount of pages beginning with \emph{the first page
of the introduction} and ending with the \emph{last page of the summary}. Hence, the
cover page, the table of contents, reference lists, and attachments do \emph{not} count
as parts of the volume. Here are some tried and tested benchmarks for the volume of
different kinds of papers and scripts:

\begin{itemize}
\item 25--100 pages for a Master thesis (see \cite{mpo08} \S 17 (5)),
\item 5--10 pages for a seminar paper (see \cite{mpo08} \S 16 (3)),
\item 15--20 pages for a lab report.
\end{itemize}

If the paper/report/thesis exceeds the given volume, it should be reconsidered whether
the conceptual formulation is too complex and, with the tutor's agreement, needs to
be reduced.


\section{The Front Page}

Every written elaboration has a front page containing the following information:

\begin{enumerate}
\item a reference to the type of elaboration (Master thesis, seminar paper, lab report),
\item for lab reports and seminar papers: Title and semester of the course,
\item title of the elaboration,
\item author's name and matriculation number,
\item date of the current version of the elaboration.
\end{enumerate}

Master theses must also contain a separate page with the annotation that the author
prepared the thesis on his own (for further information refer to the conditions of
study/MaPO).


\section{Font}

Naturally, the choice of a font is influenced by the individual preferences and style
of the author. However, we do not recommend to use the entire range of available beautiful
(or less beautiful) fonts. You will always be on the safe side with a \textbf{12 point
serif font}. Serifs are tiny dashes added to the main lines of the letters in order to
make the font more distinct and the reading process more comfortable. One of the most
widely-used serif fonts is \emph{Times Roman}, and accordingly, the respective Microsoft
variation \emph{Times New Roman}. The standard font for LaTeX documents is \emph{Computer Modern}
which was also used for this text.

In print, serif fonts are considered to feature a higher legibility than sans-serif
fonts (such as \emph{Helvetica}, \emph{Arial}, or \emph{Verdana}) and thus are especially
recommendable for long texts. Almost every newspaper, magazine, or book uses a serif font for
the actual text body.

As to the formatting, a single-column \textbf{Center justification} and single line spacing
are most favourable for the simple reason that it looks best. For the page layout we recommend
a \textbf{standard margin} (e.g. 2.5 cm) on all sides.
